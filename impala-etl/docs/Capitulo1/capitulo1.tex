%---------------------------------------------------
% Nombre: capitulo1.tex  
% 
% Texto del capitulo 1
%---------------------------------------------------

\chapter{Introducci�n}

En este documento encontramos el resultado final de la pr�ctica \textbf{ETL con Impala} \cite{impala}, enmarcada dentro de la asignatura de Big Data I del m�ster en Ciencia de Datos de la Universidad de Granada. 

\section{Problema a resolver}
\label{problema}

Se pide dise�ar un experimento de carga y extracci�n de informaci�n de una base de datos usando para ello, Impala, un motor de consultas para el procesamiento masivo en paralelo de datos almacenados en el sistema de archivos distribuido de Hadoop. La base de datos elegida, es una base de datos de quejas de usuarios de tarjeta de cr�dito en Estados Unidos, con un total de 65499 muestras con 18 caracter�sticas por cada muestra. 

\section{Objetivos}

Los objetivos de esta pr�ctica podr�an resumirse en los siguientes:

\begin{itemize}
	\item Creaci�n y carga de la base de datos 
	\item Realizar consultas que incluyan una proyecci�n, selecci�n, agrupamiento y funciones de agregado y resumen sobre los datos.
\end{itemize}

\section{Organizaci�n del trabajo} 
Tras la introducci�n al problema y los objetivos de la pr�ctica, en el cap�tulo \ref{dos} detallamos cada uno de los pasos seguidos para el dise�o del experimento de datos, junto capturas de pantalla de las salidas de las diferentes consultas. En el cap�tulo \ref{tres} se ver� una peque�a conclusi�n sobre la pr�ctica. 

\clearpage
%---------------------------------------------------